\chapter{Introduction}

\section{Background And Motivation}
In recent years, the motorbike \gls{db} industry has witnessed increasing complexity in dashboard \gls{sw} to enhance the overall riding experience for end users. This shift has necessitated advanced testing and validation techniques to ensure that the system remains safe and reliable. However, manual testing presents several limitations, including time inefficiency, high operational costs due to resource demands and susceptibility to human error. To overcome these challenges, implementing automated testing methods has become essential.

With advancements in \gls{od}, a key area within computer vision, new opportunities have emerged for automating \gls{db} \gls{sw} testing. Furthermore, \gls{dl}, a branch of \gls{ai}, has significantly improved \gls{od}, it allows real-time image processing and analysis with high accuracy and less development effort compared to traditional \gls{od} techniques. By integrating an appropriate \gls{cs} with an efficient \gls{dl}-based \gls{od} algorithm, it is possible to capture images, analyze them, extract all \gls{db} assets and their locations and transmit these data to a \gls{hil} system. The \gls{hil} system then compares the detected elements to reference values to determine whether they match.

This study focuses on the application of \gls{dl}-based \gls{od} algorithm for motorbike \gls{db} validation in an automated testing environment, aiming to enhance the efficiency of the testing department at KTM.

\section{Study Structure}
This study is structured in a way to provide a comprehensive investigation on all the key components that are necessary to develop a this automated testing environment project. It starts with this introduction chapter that outlines the motivation behind the project and the structure of the study.

Chapter 2 establishes the technical background and starts by explaining the importance of \gls{db} testing in the automotive industry, defines the testing process and comparing between manual and automated testing along with their respective requirements. Additionally, it investigates \gls{od} techniques, starting from traditional approaches and their limitations, followed by an explanation of why \gls{dl} techniques offer a more effective solution then explaining their concept of operation. Moreover, the chapter presents different \gls{dl}-based \gls{od} algorithms commonly used and their applications were introduced along with a final conclusion to guide the selection of the most suitable technique for this project. Last part of chapter 2 discuss the selection criteria of \gls{cs}s and their impact on \gls{od} models in terms of performance. 

Finally, the conclusion addresses the ability of \gls{dl} algorithms to meet the automotive industry needs in automating the \gls{db} testing process and determines which algorithms and \gls{cs} selection criteria are most suitable for this project.
